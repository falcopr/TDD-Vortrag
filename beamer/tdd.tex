\documentclass{beamer}
\usepackage[utf8x]{inputenc}
\usepackage[ngerman]{babel}
\usepackage{amsmath}
\usepackage{amsfonts}
\usepackage{amssymb}
\usepackage{graphicx}
\usepackage{subfigure}
\author{Johannes Hackel und Falco Prescher}
\title{Testgetriebene Entwicklung}

\usetheme{Ilmenau}
\useoutertheme{split}
\usecolortheme{rose}

\begin{document}

\begin{frame}
\titlepage
\end{frame}

\begin{frame}
\frametitle{Gliederung}
\tableofcontents
\end{frame}

\section{Allgemeines zur testgetriebenen Entwicklung}
\begin{frame}
\frametitle{Definition}
\end{frame}

\begin{frame}
\frametitle{Herkunft}
\end{frame}

\begin{frame}
\frametitle{Zweck}
\end{frame}

\begin{frame}
\frametitle{Prinzipien}
\end{frame}

\section{Vorgehensweise der testgetriebenen Entwicklung}
\begin{frame}
\frametitle{Vorgehensweise der testgetriebenen Entwicklung}
\end{frame}

\begin{frame}
\frametitle{Unit Tests}
\end{frame}

\begin{frame}
\frametitle{Testentwurf}
\end{frame}

\begin{frame}
\frametitle{Entwicklungszyklus}
\end{frame}

\section{Das Inversion Of Control im Zusammenhang mit Mocks}
\begin{frame}
\frametitle{Mocks}
\end{frame}

\begin{frame}
\frametitle{Inversion Of Control}
\end{frame}

\begin{frame}
\frametitle{Kombination von Inversion Of Control mit Mocks}
\end{frame}

\section{Testgetriebene Entwicklung in der Praxis}
\begin{frame}
\frametitle{siehe Beispiel}
\end{frame}

\begin{appendix}
\begin{frame}
\frametitle{Quellen}
\begin{itemize}
\item http://www.webuser.hs-furtwangen.de/kaspar/seminar0607/TestDrivenDevelopement.pdf
\item http://junit.sourceforge.net/javadoc/overview-summary.html
\item OSHEROVE, Roy: the art of UNIT TESTING - with Examples in .NET. Manning Publications Co., 2009
\end{itemize}
\end{frame}
\end{appendix}

\end{document}