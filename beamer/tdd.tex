\documentclass{beamer}
\usepackage[utf8x]{inputenc}
\usepackage[ngerman]{babel}
\usepackage{amsmath}
\usepackage{amsfonts}
\usepackage{amssymb}
\usepackage{graphicx}
\usepackage{subfigure}
\author{Johannes Hackel und Falco Prescher}
\title{Testgetriebene Entwicklung}

\usetheme{Ilmenau}
\useoutertheme{split}
\usecolortheme{rose}

\begin{document}

\begin{frame}
\titlepage
\end{frame}

\begin{frame}
\frametitle{Gliederung}
\tableofcontents
\end{frame}

\section{Allgemeines zur testgetriebenen Entwicklung}
\begin{frame}
\frametitle{Allgemeines zur testgetriebenen Entwicklung}
\end{frame}

\section{Vorgehensweise der testgetriebenen Entwicklung}
\begin{frame}
\frametitle{Entwicklungszyklus}
\begin{figure}[htbp]
\includegraphics[width=5cm]{zyklus.png}
\caption{Der Entwicklungszyklus von Software mittels testgetriebener Entwicklung}
\end{figure}
\end{frame}

\section{Das Inversion Of Control im Zusammenhang mit Mocks}
\begin{frame}
\frametitle{Mocks}
\begin{itemize}
\item Mocking - engl. für Vortäuschen
\item Erstellung eines Versuchsobjektes durch eine vorgetäuschte Implementierung von Schnittstellen oder abstrakten Klassen
\item Erzeugung des Objektes während der Laufzeit durch Mockframeworks
\item Implementierung ohne Geschäftslogik
\item Enthalten Konfigurations- und Testverifizierungsmöglichkeiten für Unit Tests
\item Mockframeworks: Mockito\footnote{http://www.code.google.com/p/mockito/} (Java) und Moq\footnote{http://www.code.google.com/p/moq/} (.NET)
\end{itemize}
\end{frame}

\begin{frame}
\frametitle{Inversion Of Control}
\begin{itemize}
\item kurz IoC - engl. für Steuerungsunkehrung
\item Aufrufen von benutzerspezifische Methoden durch das Verhalten eines Softwareframeworks
\item Dependency Injection (DI) = Spezialform des IoC\\Möglichkeit zur Registrierung Schlüssel-Wertpaaren zu einem Container in einer Startsequenz (Bootstrap)
\item Schlüssel = abstrakte Klasse oder Schnittstelle\\Wert = Methode zur Rückgabe eines Objektes (Bsp. Konstruktoren oder Methoden)
\item Inversion Of Control - Frameworks:
PicoContainer\footnote{http://www.picocontainer.codehaus.org/} (Java) und StructureMap\footnote{http://www.docs.structuremap.net/} (.NET)
\end{itemize}
\end{frame}

\begin{frame}
\frametitle{Inversion Of Control}
\begin{figure}[htbp]
\includegraphics[width=7cm]{logging_closeCoupled.jpg}
\caption{LogManager ist direkt abhängig von einer konkreten Implementierung von LogWriter}
\end{figure}
\end{frame}

\begin{frame}
\frametitle{Inversion Of Control}
\begin{figure}[htbp]
\includegraphics[width=9cm]{logging_looseCoupled.jpg}
\caption{Der Container erzeugt Instanzen und Injiziert Abhängigkeiten}
\end{figure}
\end{frame}

\begin{frame}
\frametitle{Kombination von Inversion Of Control mit Mocks}
\begin{itemize}
\item IoC - Ermöglichen des Benutzens von Mocks zur Testzeit und Produktivcode zur Produktivzeit
\item Mocks - Dadurch lose Kopplung von Programmkomponenten und Testen als einzelnes\\
\item Ohne Inversion Of Control keine Mocks zum Testen nutzbar
\item Ohne Mocks keine testgetriebene Entwicklung für große Programme möglich
\end{itemize}
\end{frame}

\section{Testgetriebene Entwicklung in der Praxis}
\begin{frame}
\frametitle{siehe Beispiel}
\end{frame}

\begin{appendix}
\begin{frame}
\frametitle{Quellen}
\begin{itemize}
\item http://www.codefest.at/post/2009/11/27/Design-Patterns-Teil-1-e28093-Inversion-of-Control-Dependency-Injection.aspx
\item http://www.code.google.com/p/mockito/
\item http://www.code.google.com/p/moq/
\item http://www.junit.sourceforge.net/javadoc/overview-summary.html
\item http://www.picocontainer.codehaus.org/
\item http://www.docs.structuremap.net/
\item http://www.webuser.hs-furtwangen.de/kaspar/seminar0607/TestDrivenDevelopement.pdf
\item OSHEROVE, Roy: the art of UNIT TESTING - with Examples in .NET. Manning Publications Co., 2009
\end{itemize}
\end{frame}
\end{appendix}

\end{document}